\documentclass{article}
\usepackage[english, russian]{babel}
\usepackage{amsfonts, amsmath}
\usepackage{geometry}
\geometry{
  top=25mm, 
}

\author{Колесова Маша, Кантимиров Игорь }
\date{}

\begin{document}

\maketitle
\noindentПлоскость $\pi$ проходит через точки A(2,1,3), B(2,4,0), C(-3,0,4). Зафиксирован репер (A, $\overrightarrow{AB}$, $\overrightarrow{AC}$).\\
\noindentНаходим координаты новых базисных векторов $\overrightarrow{AB}=(0,3,-3)$, $\overrightarrow{AC}=(-5,-1,1)$\\\\
1) \textit{Точка p \in $ \pi$ имеет в этом репере координаты (5,3). Найдите координаты точки p в стандартном репере пространства $\mathbb{R}^3$.}\\\\
Имея координаты точки в новом репере, найдем их в стандартном, пользуясь определением координат точки и разложив их по базису:\\\\
A+5$\overrightarrow{AB}$+3$\overrightarrow{AC}$=(2,1,3)+5*(0,3,-3)+3*(-5,-1,1)=(-13,13,-9)\\\\
Таким образом, получили координаты (-13,-13,-9).\\\\
2) Найдем уравнение плоскости в пространстве по 3 точкам в стандартном репере:\\
\begin{tabular}{|c c c|}
x-2 & y-1 & z-3 \\
0 & 3 & -3 \\
-5 & -1 & 1 \\
\end{tabular}
= 0\\\\
В каноническом виде получаем (x-2)*0+(y-1)*15+(z-3)*15=0\\
Или в параметрическом 
\begin{equation*}
 \begin{cases}
   x = -3 + 2$\lambda$, 
   \\
   y = 6 + 0$\lambda$,
   \\
   z = -2 + 3$\lambda$
 \end{cases}
\end{equation*}
Подставим x, y, z в уравнение плоскости, чтобы найти точку пересечения
\begin{equation*}
 \begin{cases}
   90 - 30 + 45*$\lambda$ - 60 = 0, 
   \\
   45$\lambda$ = 0,
   \\
   $\lambda$ = 0
 \end{cases}
\end{equation*}
\Rightarrow 
\begin{equation*}
 \begin{cases}
   x = -3, 
   \\
   y = 6,
   \\
   z = -2
 \end{cases}
\end{equation*}\\
Находим координаты D' = (t_1, t_2) в плоскостном репере.
(-3,6,-2) - (2,1,3) = t_1(0,3,-3) + t_2(-5,-1,1) \Rightarrow 
\begin{equation*}
 \begin{cases}
   -5t_2 = -5, 
   \\
   3t_1 - t_2 = 5,
   \\
   -5 = -3t_1 + t_2
 \end{cases}
\end{equation*}\\
\Rightarrow 3t_1 = 6, t_2 = 1
\Rightarrow D' = (2,1) \in (A, $\overrightarrow{AB}$, $\overrightarrow{AC}$).
\end{document}

